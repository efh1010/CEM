\chapter{Incidenza obliqua}

	Avendo una radiazione incidente con angolo obliquo su un intefaccia tra dielettrici diversi per studiare la trasmissione nel mezzo 2 è necessario scomporre la radiazione in propagazione trasverso elettrica e trasverso magnetica.

	\section{Angolo di incidenza}

		\begin{equation}
		\theta=arctan(\frac{E_z}{E_x})
		\end{equation}



	\section{Angolo di trasmissione}
	Si ricava , utilizzando la legge di Snell:
	\begin{equation}
	\theta_T=\arcsin(\frac{\sqrt{\epsilon_{r1}}}{\sqrt{\epsilon_{r2}}}\sin(\theta_i)
	\label{eq:leggedisnell}
	\end{equation} 
	
	\section{Condizioni limite}

		\subsection{T.I.R.}
			La condizione di trasmissione totale interna implica un coefficiente di riflessione $\Gamma$ unitario.

			\textbf{NOTA:}

			Si ricordi che in caso si TIR il fatto che non ci sia potenza trasmessa nel mezzo 2 non implica il fatto che i campi siamo nulli.
			L'intensità dei campi nel mezzo 2 possono essere calcolati attraverso la valutazione dell'onda evanescente:


		\begin{equation}
		|E|=E_0 e^{-2 \alpha x}
		\end{equation}

		con:

		\begin{equation}
		\alpha= \frac{2 \pi}{\lambda}\sqrt{1-\sin^2(\theta_t)}
		\end{equation}
		
		Si ricordi che non è possibile calcola l'angolo $\theta_t$ in quanto siamo in condizione di riflessione totale è pero possibile calcolarne il seno tramite la legge di Snell.


		\subsection{Angolo di Brewster}

			Nel caso di angolo di incidenza uguale all'angolo di Brewster\footnote{Vedi la ~\ref{eq:brewster}} abbiamo trasmissione totale nel mezzo 2, cioe $\Gamma=0$.

			Si ricordi che viene trasmessa interamente solo la componente normale della potenza incidente.

	\section{Modo trasverso elettrico}

		\subsection{Impedenza caratteristica TE}
			\begin{equation}
			\eta_{TE}=\frac{\eta_0}{\sqrt{\epsilon_r}\cos(\theta)}
			\end{equation}

		\subsection{Coefficiente di riflessione}
			Conoscendo le impedenze intrinseche posso calcolare il coefficiente di riflessione per il modo TE come:

			\begin{equation}
			\Gamma_{TE}=\frac{\eta_{TE2}-\eta_{TE1}}{\eta_{TE1}+\eta_{TE2}}
			\end{equation}
		%	 Si noti che questo coefficiente di riflessione è valido solo per la componente tangente del campo elettrico.

			 \textbf{NOTA:}
			 \\ 
			 Nello studio della propagazione di modi trasverso elettrici il coefficiente di trasmissione $\Gamma_{TE}$ si può utilizzare solo per il calcolo del campo elettrico e non del campo magnetico , il quale si potrà poi essere calcolato utilizzando la ~\ref{eq:campo-magnetico}.

		\subsection{Potenza trasmessa}

			Per calcolare la potenza trasmessa nel mezzo 2:
			\begin{itemize}
			\item Proietto la potenza incidente sull'asse normale all'intefaccia.

			\begin{equation}
			S_{inc_\perp}=S_{inc}\cos(\theta_i)
			\end{equation}

			\item Tramite il coefficiente di trasmissione calcolo la componente normale della potenza trasmessa:
			\begin{equation}
			S_{t_\perp}=S_{inc_\perp}(1-|\Gamma_{TE_\perp}|^2)
			\end{equation}

			\item Conoscendo l'angolo di trasmissione e la componente normale è possibile calcolare la potenza totale trasmessa nel mezzo 2.

			\begin{equation}
			S_{t}=\frac{S_{t_\perp}}{\cos(\theta_t)}
			\end{equation}
			\end{itemize}
		\subsection{Campo elettrico}

			Per valutare il campo elettrico in un punto dello spazio devo valutare diversi coefficienti di propagazione:
			\begin{equation}
			\gamma=j\omega\sqrt{\mu\epsilon}\end{equation}

			Da cui ricavo le due componenti:

			\begin{equation}
			\gamma_x=\gamma\sin(\theta)
			\end{equation}

			\begin{equation}
			\gamma_y=\gamma\cos(\theta)
			\end{equation}

			E' quindi possibile esprimere il campo elettrico propagante\footnote{Valido sia per modi TE che TM} nel mezzo come:
			
			\begin{equation}
			E(x,y)=E(0,0)e^-{j\gamma_x x}e^-{j\gamma_y y}
			\end{equation}
		
			\textbf{NOTA:}

			Si ricordi che studiando il modo trasverso elettrico il coefficiente di riflessione va applicato alla sola componente tangente del campo elettrico\footnote{Unica presente in modo TE.}.
			
	\section{Modo trasverso magnetico}

			\subsection{Impedenza instrinseca TM}

				\begin{equation}
					\eta_{TM}=\frac{\eta_0}{\sqrt{\epsilon_r}}\cos(\theta)
				\end{equation}

			\textbf{NOTA:}

			Si ricordi che studiando il modo trasverso magnetico il coefficiente di riflessione $\Gamma$ va utilizzato per calcolare la componente di campo elettrico trasmesso normale all'interfaccia.
			Ricavando poi con la legge di Snell l'angolo di trasmissione e conoscendo la componente normale posso trovare il campo trasmesso totale.

		\subsection{Coefficiente di riflessione}
			Conoscendo le impedenze intrinseche posso calcolare il coefficiente di riflessione per il modo TM come:

			\begin{equation}
			\Gamma_{TM}=\frac{\eta_{TM2}-\eta_{TM1}}{\eta_{TM1}+\eta_{TM2}}
			\end{equation}
		
			 
			 \textbf{ NOTA: }
			 
			 Nello studio della propagazione di modi trasverso elettrici il coefficiente di trasmissione $\Gamma_{TM}$ si può utilizzare solo per il calcolo del campo magnetico e non del campo elettrico , il quale si potrà poi essere calcolato utilizzando la ~\ref{eq:campo-elettrico}.

		\subsection{Potenza trasmessa}

			Per calcolare la potenza trasmessa nel mezzo 2:
			
			\begin{itemize}
			
			\item Proietto la potenza incidente sull'asse normale all'intefaccia.
			\begin{equation}
			S_{inc_\perp}=S_{inc}\cos(\theta_i)
			\end{equation}

			\item Tramite il coefficiente di trasmissione calcolo la componente normale della potenza trasmessa:
			\begin{equation}
			S_{t_\perp}=S_{inc_\perp}(1-|\Gamma_{TM_\perp}|^2)
			\end{equation}

			\item Conoscendo l'angolo di trasmissione e la componente normale è possibile calcolare la potenza totale trasmessa nel mezzo 2.
			\begin{equation}
			S_{t}=\frac{S_{t_\perp}}{\cos(\theta_t)}
			\end{equation}
			

			\end{itemize}
		

		\subsection{Campo magnetico}

			Per valutare il campo magnetico in un punto dello spazio devo valutare diversi coefficienti di propagazione:
			\begin{equation}
			\gamma=j\omega\sqrt{\mu\epsilon}\end{equation}

			Da cui ricavo le due componenti:

			\begin{equation}
			\gamma_x=\gamma\sin(\theta)
			\end{equation}

			\begin{equation}
			\gamma_y=\gamma\cos(\theta)
			\end{equation}

			E' quindi possibile esprimere il campo magnetico propagante\footnote{Valido sia per modi TE che TM.} nel mezzo come:
			
			\begin{equation}
			H(x,y)=H(0,0)e^-{j\gamma_x x}e^-{j\gamma_y y}
			\end{equation}
		

		\section{Onde incidenti generiche}

		In presenza di un onda non solamente TM o TE si studiano i due tipi di modi separatamente e si compongono alla fine con la somma quadratica\footnote{Ciò vale sia per le potenze che per i campi.}:

		\begin{equation}
		S_{t}=\sqrt{S_{t_{TE}}^2 +S_{t_{TM}}^2 }
		\end{equation}