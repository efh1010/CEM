
\chapter{Linee di Trasmissione}

	\section{Capacità}

	
		\subsection{Linee Coassiali}
				\begin{equation}
				C=\frac{2\pi \epsilon_0 \epsilon_r} {ln (\frac { r_{out}} {r_{in} } )}
				\end{equation}

				Se la linea è rimepita con diverse sezioni di circonferenza di dielettrici, questa puo essere modellizata come il parallelo di due capacità sommate pesandole sulla percentuale di volume occupata dai rispettivi dielettrici.
			
				Se la linea è riempita con dielettrici diversi coassiali, si puo considerare come due coassiali in serie e la capacità risulta come capacità in serie.\footnote{Si ricordi invece che il calcolo dell'induttanza rimane invariato}


		\subsection{Linee Bifilari}

			Utilizzando l'approssimazione dei conduttori sottili si può scrivere che la capacità per unità di lunghezza è data dal rapporto fra la carica per unità di lunghezza e la 
			differenza di potenziale tra i due fili\footnote{vedi la ~\ref{eq:potenzialebifilari}}.
			

			\begin{equation}
			C=\frac{\rho_l 2 \pi \epsilon_0}{\rho_l ln(\frac{d^2}{R^-R^+})}
			\label{eq:capacitabifilare}
			\end{equation}



		\subsection{Microstrisce}

	

	 		Se la linea è costituita da tratti riempiti con dielettrico di diversa natura, la linea può essere modellizzata come più capacità in parallelo calcolate come:

	 		\begin{equation}
			C=\epsilon_0 \epsilon_r\frac{w}{h}	 			
	 		\end{equation} 
	 		dove:
	 		\begin{itemize}
	 		\item $w$ è la lunghezza del settore di linea considerato.
	 		\item $h$ è la distanza tra le strisce
	 		\end{itemize}


	\section{Induttanza}

		\subsection{Linee coassiali}

			Per il calcolo dell’induttanza non è necessario tenere conto della presenza del dielettrico
			poiché questa è indipendente da esso e quindi è pari a quella in aria.
		
			\begin{equation}
				L=\frac{\mu_0 \epsilon_0}{C_0}=\frac{\mu_0 \epsilon_0 ln (\frac { r_{out}} {r_{in} } )}{2\pi \epsilon_0}
			\end{equation}



		\subsection{Linee bifilari}

		Ricordando la ~\ref{eq:capacitabifilare} e la relazione $L=\frac{\mu_0 \epsilon_0}{C_0}$ è immediato calcolare l'induttanza della linea.

		\subsection{Microstrisce}
			
			Per il calcolo dell’induttanza, analogamente a quanto detto per la linea coassiale, non è necessario tenere conto della presenza del dielettrico
			poiché questa è indipendente da esso e quindi è pari a quella in aria.
		
			\begin{equation}
				L=\frac{\mu_0 \epsilon_0}{C_0}=\mu_0 \frac{h}{w}
			\end{equation}
	


	\section{Impedenza caratteristica}

			L'impedenza caratteristica di una linea di trasmissione è il rapporto dei moduli della tensione e della corrente che si propagano in una linea distribuita in una singola direzione, in assenza di riflessioni.

			\begin{equation}
				Z_c=\sqrt{\frac{L}{C}}
			\end{equation}

	\section{Spessore pelle}
			Lo spessore pelle indica la profondità di penetrazione del campo in un conduttore.
			\begin{equation}
			\delta=\frac{1}{	\sqrt{\pi \sigma f \mu_o}}
			\end{equation}

	\section{Velocità di fase}
			 La velocità di fase può essere visualizzata come la velocità di propagazione di una cresta dell'onda ma non coincide necessariamente con la velocità di propagazione di un segnale (che è più propriamente descritta dalla velocità di gruppo) e quindi può essere più alta della velocità della luce senza violare la relatività ristretta.
			\begin{equation}
				v_f=\frac{1}{\sqrt{LC}}
			\end{equation}

	\section{Resistenza}

		\subsection{Linee Coassiali}

			La resistenza per unità di lunghezza si calcola come:

			\begin{equation}R=\frac{1}{\sigma \delta p}=\frac{1}{\sigma \delta 2 \pi}(\frac{1}{r_{out}}+\frac{1}{r_{in}})\end{equation}
			dove:
			
			\begin{itemize}
			\item $\sigma$ è la conducibilità del conduttore.
			\item $\delta$ è lo spessore pelle.
			\item $p$ è il perimetro della sezione nel conduttore interessata dal passaggio di corrente.
			\end{itemize}

		\subsection{Linee bifilari}

				La resistenza per unità di lunghezza per una linea bifilare si calcola come:

				\begin{equation}R=\frac{2w}{\sigma \delta}\end{equation}
				dove:
				
				\begin{itemize}
				\item $\sigma$ è la conducibilità del conduttore
				\item $\delta$ è lo spessore pelle
				\item $w$ è la lunghezza della linea
				\end{itemize}

		\subsection{Microstrisce}

			Analogamente ai casi precedenti si calcola le resistenza considerando il solo spessore di conduttore ricavato con lo spessore pelle e la larghezza della microstriscia.\footnote{Il calcolo è quindi analogo alla linea coassiale sostiuendo il perimetro con la larghezza della microstriscia.}

	\section{Conduttanza per unità di lunghezza}

			\begin{equation}
			g=\frac{\sigma C}{\epsilon}
			\end{equation}

	%g=\frac{\sigma_d * Area}{h}	= \sigma_d \frac{w}{h}




	\section{Costante di attenuazione dovuta ai conduttori}
	
			La costante di attenuazione dovuta alla conducibilità finita dei conduttori si trova da:
			\begin{equation}
			\alpha_c=\frac{R}{2Z_c}
			\end{equation}
		
	\section{Costante di attenuazione dovuta al dielettrico}
	
			L'attenuazione dovuta alle perdite\footnote{per unità di lunghezza.} nel dielettrico è conseguenza di una conducibiltà non nulla di materiali non ideali.

			\begin{equation}
			\alpha_d=\frac{g}{2Y_c}
			\end{equation}

	\section{Attenuazione totale}

			\begin{equation}
			A=\alpha L
			\end{equation}
			Dove:
			\begin{itemize}
			\item L è la lunghezza totale della linea.
			\item $\alpha=\alpha_d+\alpha_c$ è la costante di attenuazione totale che tiene conto sia delle perdite nel dielettrico che nei conduttori.
			\end{itemize}

	\section{Conversione Np-dB}

	\begin{equation}
	1\frac{Np}{m}=8686\frac{dB}{km}=8,686\frac{dB}{m}
	\end{equation}
	
	\section{Potenza associata all'onda progressiva}

			\begin{equation}
			P=\frac{1}{2}\frac{|V^+|^2}{Z_0}=\frac{1}{2}|I^+|^2 Z_0
			\end{equation}

	\section{Potenza dissipata}

			\begin{equation}
			P_{diss,COND}=\frac{1}{2}|I_{max}|^2 r
			\end{equation}

			\begin{equation}
			P_{diss,DIEL}=\frac{1}{2} g V^2
			\end{equation}

	\section{Campo magnetico massimo}
	
 			Ricordando la legge di Ampere conoscendo la corrente massima si può calcolare i campo magnetico massimo come:

			\begin{equation}
			2 \pi a |H_{max}| = |I_{max}|
			\end{equation}

	\section{Costante di fase}
			%		    \subsection{Microstrisce}
			\begin{equation}
				\beta=\omega \sqrt{LC}=2 \pi f \sqrt{LC}
			\end{equation}
