		
\chapter{Propagazione in multistrato}

	\section{Impedenza intrinseca}

		L'impedenza intrinseca per la propagazione in un mezzo dielettrico\footnote{In assenza di perdite} si calcola come:

		\begin{equation}
		\eta=\frac{\eta_0}{\sqrt{\epsilon_r}}
		\end{equation}

		Dove $\eta_0=377\Omega$ è l'impedenza caratteristica del vuoto.

		Se il dielettrico presenta una conducibilità equivalente non nulla l'impedenza instrinseca risulta:

		\begin{equation}
		\eta=\frac{j\omega\mu}{\sqrt{-\omega^2\mu\epsilon+j\omega\mu\sigma_{eq}}} [  \Omega  ]
		\end{equation}
		Dove la conducibilità equivalente si calcola come:

		\begin{equation}
		\sigma_{eq}=\omega \epsilon''
		\end{equation}
		Risulta $\epsilon=\epsilon'+j\epsilon''$.
		
		I coefficienti $\epsilon'$ e $\epsilon''$ vengono dati tramite il parametro\footnote{In cui si ricorda che la $\delta$ non ha nulla a che fare con lo spessore pelle.} $tan\delta=\frac{\epsilon''}{\epsilon'}$.

	\section{Lunghezza d'onda in mezzi dielettrici}
		Una radiazione che si propoaga in un mezzo con constante dielettrica diversa da quella del vuoto subisce una variazione della lunghezza d'onda:

		\begin{equation} 
		\lambda = \frac {\lambda_0} {\sqrt {\epsilon_r} }
		\end{equation} 

	\section{Equivalente in linea di trasmissione}

		
		La propagazione in un mezzo multistrato può essere modellizata come una linea di trasmissione in cui i vari spezzoni di linea hanno impedenza caratteristica uguale all'impedenza intrinseca del mezzo in cui si propagano.	
		Si puo quindi studiare il problema con l'utilizzo della carta di Smith.

	\section{Condizioni al contorno per il campo elettrico}

		Nell'interfaccia tra mezzi con costante dielettrica diversa abbiamo che:
		\begin{equation}
		E_{t1}=E_{t2} 
		\end{equation}
		cioé $\epsilon_1 D_{t1} = \epsilon_2 D_{t2}$.
		Mentre per la componente normale si ha:
		\begin{equation}
		D_{n1} = D_{n2}
		\end{equation}
		cioè:
		\begin{equation}
		E_{n1}\epsilon_1=E_{n2}\epsilon_2
		\end{equation}

	\section{Condizioni al contorno per il campo magnetico}
		Nell'interfaccia tra mezzi con permeabilità magnetica diversa si ha:
		\begin{equation}
		B_{n1}=B_{n2} 
		\end{equation}
		cioé $\mu_1 H_{n1} = \mu_2 H_{n2}$.
		Mentre si conserva la componente tangenziale\footnote{Assumendo nulla la densità superficiale di corrente all'intefaccia. Tenendone conto risulterebbe: $J_s=H_{t1}-H_{t2}$}.
		\begin{equation}
		H_{t1}=H_{t2}
		\end{equation}

	\section{Campo elettrico e campo magnetico}

	Conoscendo la densità di potenza propagante possiamo calcolare il modulo di campo elettrico e magnetico come:

	\begin{equation}
	S=\frac{E^2}{2\eta_{mezzo}}=\frac{H^2}{2}\eta_{mezzo}
	\end{equation}

	\section{Vettore di Poynting}
		Per una radiazione propagante in un mezzo possiamo calcolare il valore medio del vettore di Poynting come:
		\begin{equation}
		\overline{S_{avg}}=\frac{A^2}{2\eta}
		\end{equation}
		Dove A è l'ampiezza del campo elettrico.
		E' possibile calcolare anche il valore istantaneo che risulta:

		\begin{equation}
		S_{ist}=\frac{A^2}{\eta}\cos^2(\omega t)
		\end{equation}
	
	\section{Legge di snell}

	La legge di snell viene utilizzata per studiare il caso semplificato di impedenza normale.

	\begin{equation}
	n_i \sin(\theta_i) = n_t \sin(\theta_t)
	\end{equation}

	Con $n_i=\sqrt{\epsilon_i}$ e $n_t=\sqrt{\epsilon_t}$.
