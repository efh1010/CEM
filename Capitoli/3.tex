
\chapter{Sorgenti di campo magnetico}

	\section{Spira}
			Il campo magnetico generato da una spira circolare nella direzione assiale\footnote{Asse z} è:

			\begin{equation}
			B=\frac{\mu IR^2}{2(z^2+R^2)^{\frac{3}{2}}}
			\end{equation}

		\subsection{Legge di Faraday-Neumann-Lenz}

		\begin{equation}
		f.e.m.=Area* \mu \frac{dH}{dt} cos(\theta) 
		\end{equation}
		Dove $\theta$ è l'angolo compreso tra la direzione del campo magnetico e il vettore normale alla spira.
	\section{Solenoide}
		\subsection{Campo magnetico in un solenoide}
			Il campo magnetico al centro di un solenoide\footnote{Assumento il solenoide riempito con un materiale con permeabilità magnetica relativa $\mu_r$} è:

			\begin{equation}
			B=\mu_0\mu_rNI
			\end{equation}

			Dove:

			\begin{itemize}
			\item $N$ è il numero di spire del solenoide.
			\item $I$ è la corrente che scorre nel solenoide.
			\end{itemize}

		\subsection{Induttanza solenoide}
				\begin{equation} 
				L=\mu \frac{N^2S}{l}
				\label{eq:induttanza-solenoide}
				\end{equation}

	\section{Conduttore cilindrico}
		Dato un conduttore cilindrico di raggio $R$ e attraversato da un dendità di corrente $J$ possiamo calcolare il campo magnetico in tutto la spazio applicando la legge di ampere e risulta:

		\begin{itemize}
		\item Soluzione interna: $H_\phi(R)=\frac{jR}{2}$ 
		\item Soluzione esterna: $H_\phi(R)=\frac{I}{2\pi R}$

		\end{itemize}
