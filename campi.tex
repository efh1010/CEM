%configurazioni	
	\documentclass[10pt,a4paper]{report}
	\usepackage{graphicx}
	\usepackage[utf8]{inputenc}
	\usepackage{amssymb,amsmath}
	\usepackage{geometry}
	\usepackage[italian]{babel}  
	\usepackage[bookmarks=true]{hyperref}
	\usepackage{bookmark}
	\geometry{letterpaper}	

\hypersetup{
	pdftitle={Campi elettromagnetici},%
	pdfauthor={Edoardo Contini},%
	pdfsubject={Campi Elettromagnetici},%
	pdfkeywords={},%
	colorlinks=true,%
	linkcolor=blue,%
	linktocpage=true,%
	pageanchor=true
}


\begin{document}


\begin{titlepage}

	\centering
	{\scshape\huge\textbf Politecnico di Milano \par}
	\vspace{0.5cm}

	\includegraphics[width=0.15\textwidth]{logo.png}\par\vspace{0.2cm}
	
	{\scshape\small Facoltà di ingegneria\par}
	{\scshape\small Dipartimento di elettronica e informazione\par}
	\vspace{1.5cm}
	{\huge\bfseries Campi Elettromagnetici\par}
	\vspace{1.5cm}
	{\scshape Appunti\par}
	{\scshape\small 2015-2016 \par}
	\vspace{2cm}
	%{\Large\itshape Edoardo Contini\par}
	\vfill
    \raggedleft{Professore: \\ \textit{Michele D'Amico}}	
    \\[1cm]
	
	\raggedright
    {Studente:\\ \textit{Edoardo Contini}
	
	}\vfill
	\raggedleft
	{\large \today\par}
	
	%\raggedright
	%\textit{La risposta è dentro di voi (ma è sbagliata)}.
	%{\\Corrado Guzzanti}
	\end{titlepage}

%\title{Optoelettronica}
%\author{Edoardo Contini}
%\date{}
	



%\maketitle
\tableofcontents
\pdfbookmark{\contentsname}{Indice}

\chapter{Distribuzioni di carica}

	\section{Conduttori Coassiali}
		\subsection{Tensione tra i conduttori}

		La tensione tra due conduttori coassiali riempiti con dielettrico , sui quali è distribuita una densità di carica superficiale $\rho_l$ opposta si calcola come:
		 \begin{equation}
		 V= \frac {\rho_l}  {2 \pi \epsilon_0} ln ( \frac { r_{out} } { r_{in} } )
		 \end{equation}
	
	\section{Conduttori cilindrici affiancati}

		La tensione tra due conduttori cilindrici affiancati su cui è distribuita una densità di carica $\rho_l$ opposta risulta:

		\begin{equation}
		V=\frac{\rho_l}{2 \pi \epsilon_0}{ln(\frac{d^2}{R^+R^-})}
		\end{equation}

		La capacità risulta invece:

		\begin{equation}
		C=\frac{2 \pi \epsilon_0}{ln(\frac{d^2}{R^+R^-})}
		\end{equation}

\chapter{Metodo delle immagini}	

\section{Piano conduttore}

Per annulare il potenziale su un piano conduttore pongo un carica immagine simmetricamente dalla parte opposta rispetto alla carica.

\section{Linea di trasmissione}

Esiste una soluzione unica se l'angolo tra i due piani coduttori della linea è$\frac{\pi}{n}$ con n intero.
Verificata questa condizione dovrò porre $2n-1$ cariche immagine simmetricamente rispetto al sistema.

\section{Sfera}

Per annulare il potenziale su una sfera all esterno della quale è posta una carica devo porre una carica immagine a distanza $d'$ dal centro lungo la congiungente tra il centro e la carica esterna tali che:

\[
d'=\frac{R^2}{d}
\]


\chapter{Sorgenti di campo magnetico}

	\section{Spira}
			Il campo magnetico generato da una spira circolare nella direzione assiale\footnote{z} è:

			\begin{equation}
			B=\frac{\mu IR^2}{2(z^2+R^2)^{\frac{3}{2}}}
			\end{equation}

	\section{Solenoide}
		\subsection{Campo magnetico in un solenoide}
			Il campo magnetico al centro di un solenoide\footnote{Assumento il solenoide riempito con un materiale con permeabilità magnetica relativa $\mu_r$} è:

			\begin{equation}
			B=\mu_0\mu_rNI
			\end{equation}

			Dove:

			\begin{itemize}
			\item $N$ è il numero di spire del solenoide.
			\item $I$ è la corrente che scorre nel solenoide.
			\end{itemize}

		\subsection{Induttanza solenoide}
				\begin{equation} 
				L=\mu \frac{N^2S}{L}
				\label{eq:induttanza-solenoide}
				\end{equation}

	\section{Conduttore cilindrico}
		Dato un conduttore cilindrico di raggio $R$ e attraversato da un dendità di corrente $J$ possiamo calcolare il campo magnetico in tutto la spazio applicando la legge di ampere e risulta:

		\begin{itemize}
		\item Soluzione interna: $H_\phi(R)=\frac{jR}{2}$ 
		\item Soluzione esterna: $H_\phi(R)=\frac{I}{2\pi R}$

		\end{itemize}

\chapter{Linee di Trasmissione}

	\section{Capacità}

	
		\subsection{Linee Coassiali}
				\begin{equation}
				C=\frac{2\pi \epsilon_0 \epsilon_r} {ln (\frac { r_{out}} {r_{in} } )}
				\end{equation}

				Se la linea è rimepita con diverse sezioni di circonferenza di dielettrici, questa puo essere modellizata come il parallelo di due capacità sommate pesandole sulla percentuale di volume occupata dai rispettivi dielettrici.
			
				Se la linea è riempita con dielettrici diversi coassiali, si puo considerare come due coassiali in serie e la capacità risulta come capacità in serie.\footnote{Si ricordi invece che il calcolo dell'induttanza rimane invariato}


		\subsection{Linee Bifilari}

			Utilizzando l'approssimazione dei conduttori sottili si può scrivere che la capacità per unità di lunghezza è data dal rapporto fra la carica per unità di lunghezza e la 
			differenza di potenziale tra i due fili.
			

			\begin{equation}
			C=\frac{\rho_l 2 \pi \epsilon_0}{\rho_l ln(\frac{d^2}{R^-R^+})}
			\end{equation}



		\subsection{Microstrisce}

	

	 		Se la linea è costituita da tratti riempiti con dielettrico di diversa natura, la linea puo essere modellizzata come tre capacità in parallelo calcolate come:

	 		\begin{equation}
			C=\epsilon_0 \epsilon_r\frac{w}{h}	 			
	 		\end{equation} 
	 		dove:
	 		\begin{itemize}
	 		\item $w$ è la lunghezza del settore di linea considerato
	 		\item $h$ è la distanza tra le strisce
	 		\end{itemize}


	\section{Induttanza}

		\subsection{Linee coassiali}

			Per il calcolo dell’induttanza non è necessario tenere conto della presenza del dielettrico
			poiché questa è indipendente da esso e quindi è pari a quella in aria.
		
			\begin{equation}
				L=\frac{\mu_0 \epsilon_0}{C_0}=\frac{\mu_0 \epsilon_0 ln (\frac { r_{out}} {r_{in} } )}{2\pi \epsilon_0}
			\end{equation}



		\subsection{Linee bifilari}

		\subsection{Microstrisce}
			
			Per il calcolo dell’induttanza, analogamente a quanto detto per la linea coassiale, non è necessario tenere conto della presenza del dielettrico
			poiché questa è indipendente da esso e quindi è pari a quella in aria.
		
			\begin{equation}
				L=\frac{\mu_0 \epsilon_0}{C_0}=\mu_0 \frac{h}{w}
			\end{equation}
	


	\section{Impedenza caratteristica}

			L'impedenza caratteristica di una linea di trasmissione è il rapporto dei moduli della tensione e della corrente che si propagano in una linea distribuita in una singola direzione, in assenza di riflessioni.

			\begin{equation}
				Z_c=\sqrt{\frac{L}{C}}
			\end{equation}

	\section{Spessore pelle}
			Lo spessore pelle indica la profondità di penetrazione del campo in un conduttore.
			\begin{equation}
			\delta=\frac{1}{	\sqrt{\pi \sigma f \mu_o}}
			\end{equation}

	\section{Velocità di fase}
			 La velocità di fase può essere visualizzata come la velocità di propagazione di una cresta dell'onda ma non coincide necessariamente con la velocità di propagazione di un segnale (che è più propriamente descritta dalla velocità di gruppo) e quindi può essere più alta della velocità della luce senza violare la relatività ristretta.
			\begin{equation}
				v_f=\frac{1}{\sqrt{LC}}
			\end{equation}

	\section{Resistenza}

		\subsection{Linee Coassiali}

			La reistenza per unità di lunghezza si calcola come:

			\begin{equation}R=\frac{1}{\sigma \delta p}=\frac{1}{\sigma \delta 2 \pi}(\frac{1}{r_{out}}+\frac{1}{r_{in}})\end{equation}
			dove:
			
			\begin{itemize}
			\item $\sigma$ è la conducibilità del conduttore
			\item $\delta$ è lo spessore pelle
			\item $p$ è il perimetro della sezione nel conduttore interessata dal passaggio di corrente
			\end{itemize}

		\subsection{Linee bifilari}

				La resistenza per unità di lunghezza per una linea bifilare si calcola come:

				\begin{equation}R=\frac{2w}{\sigma \delta}\end{equation}
				dove:
				
				\begin{itemize}
				\item $\sigma$ è la conducibilità del conduttore
				\item $\delta$ è lo spessore pelle
				\item $w$ è la lunghezza della linea
				\end{itemize}

		\subsection{Microstrisce}


		\section{Conduttanza per unità di lunghezza}

			\begin{equation}
			g=\frac{\sigma C}{\epsilon}
			\end{equation}

	%g=\frac{\sigma_d * Area}{h}	= \sigma_d \frac{w}{h}




	\section{Costante di attenuazione dovuta ai conduttori}
	
			La costante di attenuazione dovuta alla conducibilità finita dei conduttori si trova da:
			\begin{equation}
			\alpha_c=\frac{R}{2Z_c}
			\end{equation}
		
	\section{Costante di attenuazione dovuta al dielettrico}
	
			L'attenuazione dovuta alle perdite\footnote{per unità di lunghezza} nel dielettrico è dovuta ad una conducibiltà non nulla di materiali non ideali.

			\begin{equation}
			\alpha_d=\frac{g}{2Y_c}
			\end{equation}

	\section{Attenuazione totale}

			\begin{equation}
			A=\alpha L
			\end{equation}
			Dove:
			\begin{itemize}
			\item L è la lunghezza totale della linea.
			\item $\alpha=\alpha_d+\alpha_c$ è òa costante di attenuazione totale che tiene conto sia delle perdite nel dielettrico che nei conduttori.
			\end{itemize}


	\section{Potenza associata all'onda progressiva}

			\begin{equation}
			P=\frac{1}{2}\frac{|V^+|^2}{Z_0}=\frac{1}{2}|I^+|^2 Z_0
			\end{equation}

	\section{Potenza dissipata}

			\begin{equation}
			P_{diss,COND}=\frac{1}{2}|I_{max}|^2 r
			\end{equation}

			\begin{equation}
			P_{diss,DIEL}=\frac{1}{2} g V^2
			\end{equation}

	\section{Campo magnetico massimo}
	
 			Ricordando la legge di Ampere conoscendo la corrente massima si può calcolare i campo magnetico massimo come:

			\begin{equation}
			2 \pi a |H_{max}| = |I_{max}|
			\end{equation}


	\section{Campo elettrico massimo}


	\section{Costante di fase}
			%		    \subsection{Microstrisce}
			\begin{equation}
				\beta=\omega \sqrt{LC}=2 \pi f \sqrt{LC}
			\end{equation}
		
\chapter{Propagazione in multistrato}

	\section{Impedenza intrinseca}

		L'impedenza intrinseca per la propagazione in un mezzo dielettrico\footnote{In assenza di perdite} si calcola come:

		\begin{equation}
		\eta=\frac{\eta_0}{\sqrt{\epsilon_r}}
		\end{equation}

		Dove $\eta_0=377\Omega$ è l'impedenza caratteristica del vuoto.

		Se il dielettrico prensenta una conducibilità equivalente non nulla l'impedenza instrinseca risulta:

		\begin{equation}
		\eta=\frac{j\omega\mu}{\sqrt{-\omega^2\mu\epsilon+j\omega\mu\sigma_{eq}}} [  \Omega  ]
		\end{equation}
		Dove la conducibilità equivalente si calcola come:

		\begin{equation}
		\sigma_{eq}=\omega \epsilon''
		\end{equation}
		Risulta $\epsilon=\epsilon'+j\epsilon''$.
		
		I coefficienti $\epsilon'$ e $\epsilon''$ vengono dati tramite il parametro $tan\delta=\frac{\epsilon''}{\epsilon'}$.

	\section{Lunghezza d'onda in mezzi dielettrici}
		Una radiazione che si propoaga in un mezzo con constante dielettrica diversa da quella del vuoto subisce una variazione della lunghezza d'onda:

		\begin{equation} 
		\lambda = \frac {\lambda_0} {\sqrt {\epsilon_r} }
		\end{equation} 

	\section{Equivalente in linea di trasmissione}

		
		La propagazione in un mezzo multistrato come nella può essere modellizata come una linea di trasmissione in cui i vari spezzoni di linea hanno impedenza caratteristica uguale a quella del mezzo in cui si propagano.	
		Si puo quindi studiare il problema con l'utilizzo della carta di Smith.

	\section{Condizioni al contorno per il campo elettrico}

	\section{Condizioni al contorno per il campo magnetico}
		Nell interfaccia tra mezzi con permeabilità magnetica diversa:
		\begin{equation}
		B_{n1}=B_{n2} 
		\end{equation}
		cioé $\mu_1 H_{n1} = \mu_2 H_{n2}$.
		Mentre si conserva la componente tangenziale\footnote{Assumendo nulla la densità superficiale di corrente all'intefaccia. Tenendone conto risulta: $J_s=H_{t1}-H_{t2}$}.
		\begin{equation}
		H_{t1}=H_{t2}
		\end{equation}

	\section{Vettore di Poynting}
		Per una radiazione propagante in un mezzo possiamo calcolare il valore medio del vettore di Poynting come:
		\begin{equation}
		\overline{S_{avg}}=\frac{A^2}{2\eta}
		\end{equation}
		Dove A è l'ampiezza del campo elettrico.
		E' possibile calcolare anche il valore instantaneo che risulta:

		\begin{equation}
		S_{ist}=\frac{A^2}{\eta}\cos^2(\omega t)
		\end{equation}
	
	\section{Legge di snell}

	La legge di snell viene utilizzata per studiare il caso semplificato di impedenza normale.


\chapter{Incidenza obliqua}

	Avendo una radiazione incidente con angolo ubliquo su un intefaccia tra dielettrici diversi per studiare la trasmissione nel mezzo 2 è necessario scomporre la radiazione in propagazione trasverso elettrica e trasverso magnetica.

	\section{Angolo di trasmissione}

	\begin{equation}
	\theta_T=\arcsin(\frac{\sqrt{\epsilon_{r1}}}{\sqrt{\epsilon_{r1}}}\sin(\theta_i)
	\end{equation} 
	

	\section{Modo trasverso elettrico}

		\subsection{Impedenza caratteristica TE}
			\begin{equation}
			\eta_{TE}=\frac{\eta_0}{\sqrt{\epsilon_r}\cos(\theta)}
			\end{equation}

		\subsection{Coefficiente di riflessione}
			Conoscendo le impedenze intrinseche posso calcolare il coefficiente di riflessione per il modo TE come:

			\[
			\Gamma_{TE}=\frac{\eta_{TE1}-\eta_{TE1}}{\eta_{TE1}+\eta_{TE1}}
			\]
			 Si noti che questo coefficiente di riflessione è valido solo per la componente normale del campo elettrica.

			 \textbf{NOTA:}
			 \\ 
			 Nello studio della propagazione di modi trasverso elettrici il coefficiente di trasmissione $\Gamma_{TE}$ si può utilizzare solo per il calcolo del campo elettrico e non del campo magnetico , il quale si potrà poi essere calcolato utilizzando la ~\ref{eq:campo-magnetico}.

		\subsection{Potenza trasmessa}

			Per calcolare la potenza trasmessa nel mezzo 2:
			\begin{itemize}
			\item Proietto la potenza incidente sull'asse normale all'intefaccia.

			\begin{equation}
			S_{inc_\perp}=S_{inc}\cos(\theta_i)
			\end{equation}

			\item Tramite il coefficiente di trasmissione calcolo la componente normale della potenza trasmessa:
			\begin{equation}
			S_{t_\perp}=S_{inc_\perp}(1-|\Gamma_{TE_\perp}|^2)
			\end{equation}

			\item Conoscendo l'angolo di trasmissione e la componente normale è possibile calcolare la potenza totale trasmessa nel mezzo 2.

			\begin{equation}
			S_{t}=\frac{S_{t_\perp}}{\cos(\theta_t)}
			\end{equation}
			\end{itemize}
		\subsection{Campo elettrico}

			Per valutare il campo elettrico in un punto dello spazio devo valutare diversi coefficienti di propagazione:
			\[
			\gamma=j\omega\sqrt{\mu\epsilon}\]

			Da cui ricavo le due componenti:

			\[
			\gamma_x=\gamma\sin(\theta)
			\]

			\[
			\gamma_y=\gamma\cos(\theta)
			\]

			E' quindi possibile esprimere il campo elettrico propagante\footnote{Valido sia per modi TE che TM} nel mezzo come:
			
			\[
			E(x,y)=E(0,0)e^-{j\gamma_x x}e^-{j\gamma_y y}
			\]
		

	\section{Modo trasverso magnetico}

			\subsection{Impedenza instrinseca TM}

				\begin{equation}
					\eta_{TM}=\frac{\eta_0}{\sqrt{\epsilon_r}}\cos(\theta)
				\end{equation}


		\subsection{Coefficiente di riflessione}
			Conoscendo le impedenze intrinseche posso calcolare il coefficiente di riflessione per il modo TM come:

			\[
			\Gamma_{TM}=\frac{\eta_{TM1}-\eta_{TM1}}{\eta_{TM1}+\eta_{TM1}}
			\]
			 Si noti che questo coefficiente di riflessione è valido solo per la componente normale del campo elettrica.
			 \\
			 \textbf{NOTA:}
			 \\ 
			 Nello studio della propagazione di modi trasverso elettrici il coefficiente di trasmissione $\Gamma_{TM}$ si può utilizzare solo per il calcolo del campo magnetico e non del campo elettrico , il quale si potrà poi essere calcolato utilizzando la ~\ref{eq:campo-elettrico}.

		\subsection{Potenza trasmessa}

			Per calcolare la potenza trasmessa nel mezzo 2:
			
			\begin{itemize}
			
			\item Proietto la potenza incidente sull'asse normale all'intefaccia.
			\begin{equation}
			S_{inc_\perp}=S_{inc}\cos(\theta_i)
			\end{equation}

			\item Tramite il coefficiente di trasmissione calcolo la componente normale della potenza trasmessa:
			\begin{equation}
			S_{t_\perp}=S_{inc_\perp}(1-|\Gamma_{TM_\perp}|^2)
			\end{equation}

			\item Conoscendo l'angolo di trasmissione e la componente normale è possibile calcolare la potenza totale trasmessa nel mezzo 2.
			\begin{equation}
			S_{t}=\frac{S_{t_\perp}}{\cos(\theta_t)}
			\end{equation}
			

			\end{itemize}
		

		\subsection{Campo magnetico}

			Per valutare il campo magnetico in un punto dello spazio devo valutare diversi coefficienti di propagazione:
			\[
			\gamma=j\omega\sqrt{\mu\epsilon}\]

			Da cui ricavo le due componenti:

			\[
			\gamma_x=\gamma\sin(\theta)
			\]

			\[
			\gamma_y=\gamma\cos(\theta)
			\]

			E' quindi possibile esprimere il campo magnetico propagante\footnote{Valido sia per modi TE che TM} nel mezzo come:
			
			\[
			H(x,y)=H(0,0)e^-{j\gamma_x x}e^-{j\gamma_y y}
			\]
		

		\section{Onde incidenti generiche}

		In presenza di un onda non solamente TM o TE si studiano i due tipi di modi separatamente e si compongono alla fine con la somma quadratica\footnote{Ciò vale sia per le potenze che per i campi}:

		\[
		S_{t}=\sqrt{S_{t_{TE}}^2 +S_{t_{TM}}^2 }
		\]

\chapter{Adattamento}

	\section{Adattatore con nutralizzatore}

		L'adattatore con neutralizzatore è un tratto di linea da inserire tra il carico e la fine dell'adattatore lambda quarti ed è utilizzato e dimensionato in modo che porti all'interfaccia del $\frac{\lambda}{4}$ un impendenza reale pura.

	\section{Adattatore lambda quarti}
		
		L'adattatore adattatore lambda quarti è un tratto di linea con impedenza caratteristica e lunghezza da dimensionare che viene inserito per adattare il carico alla linea.

		\subsection{Impedenza caratteristica}
				Muovendosi sulla carta di smith è necessario trovare le impedenze viste ad entrambe le interfacce dell'adattore.

				\begin{equation}
				\eta_{\frac{\lambda}{4}}=\sqrt{\eta_{dx}\eta_{sx}}
				\end{equation}

		\subsection{Lunghezza della linea}
				Imponendo le condizioni di interferenza distruttiva tra le onde riflesse alle 2 interfaccie si trova che la lunghezza del tratto di linea da inserire deve essere un quarto della lunghezza d'onda normalizzata all'impedenza caratteristica determinata.
				\begin{equation}
				L_s=\frac{\lambda_0}{4\eta_{\frac{\lambda}{4}}}
				\end{equation}

	\section{Stub parallelo}

		Dimendionando l'adattamento con stub parallelo è necessario lavorare con la carta delle ammettenze\footnote{Si ricordi che nella carta delle ammettenze il corto circuito si trova a destra}, ogni impedenza $Z$ va quindi trasformata in ammettenza come: $Y=\frac{1}{Z}$

		\subsection{Corto circuito}
				

			Per l'adattamento con stub parallelo trovala l'ammettenza del carico $Y_L=\frac{1}{Z_L}$ la normalizzo per l'ammettenza caratteristica del tratto di linea attraverso cui mi sto muovendo  $Y_0=\frac{1}{Z_0}$ verso il generatore come: $Y_{\overline{LL}}=\frac{Y_L}{Y_O}$.

			A questo punto riportando $Y_{\overline{LL}}$ sulla carta delle ammetenze , ruoto fino ad individuale i punti di intersezione con la circonferenza unitaria

			Indivuduata la parte immaginaria dell'impedenza , partendo dal corto circuito ruoto sulla circonferenza esterna della carta di smith\footnote{circonferenza a parte reale} ruoto fino al punto a parte immaginaria opposta rispetto a quella dell'impedenza dovuta al carico e determino cosi la lunghezza dello stub.

		\subsection{Circuito aperto}

		\subsection{Doppio stub}
	
	\section{Stub serie}

	 	Dimensionando l'addatamento con stub serie è necessadio unare la carta si delle impedenze.

		\subsection{Corto circuito}


		\subsection{Circuito aperto}

	\section{Adattamento con carico reattivo}

		Analogamente a quanto visto per gli adattatori a neutralizzazione e stub parallelo e serie, un altro metodo di adattamento del carico consiste nell'inserimento in linea di un carico reattivo\footnote{Capacità o induttanza}.

		Analogamente al discorso fatto per gli stub il carico reattivo puuò essere aggiunto in serie o in parallelo

		\subsection{Serie}

		\subsection{Parallelo}

\chapter{Propagazione in linea}



	\section{Potenza}
		In caso di adattamento abbiamo che tutta la potenza disponibile è trasferita al carico e in particolare vale:
			
			\begin{equation}
			P=(\frac{V_g}{2})^2\frac{1}{2Z}=(\frac{V_g}{2})^2\frac{1}{2}Re\{Y_L\}
			\end{equation}

			La potenza trasmessa al carico può anche essere calcolata tramite il vettore di pointing e il coefficiente di riflessione:

			\begin{equation}
			S_{tr}=S_{inc}(1-|\Gamma|^2)
			\end{equation}

			Analogo discorso si può fare con la potenza.
			La potenza trasferita al carico risulta essere:

			\begin{equation}
			P_{load}=P_{inc}(1-|\Gamma|^2)
			\end{equation}
		
	\section{Tensione e corrente sulla linea}

	\section{Attenuazione}

		Se in linea abbiamo delle perdite e conosciamo il coefficiente di attenuazione $\alpha$, sappiamo che la potenza si propaga come:
		\begin{equation}
		P(x)=e^{-\alpha}x
		\end{equation}

		Si ricordi come in presenza di perdite in linea la potenza erogata risulta uguale alla somma della potenza dissipata lungo la linea e alla potenza trasferita al carico:

		\begin{equation}
		P_{erogata}=P_{diss}+P_{load}
		\end{equation}

	\section{Rapporto d'onda stazionaria}

	\begin{equation}
	ROS=\frac{|V_{max}|}{|V_{min}|}=\frac{|V_o^+|(1+|\Gamma|)}{|V_o^+|(1-|\Gamma|)}
	\end{equation}

\chapter{Guide d'onda}
		
	\section{Modo trasverso elettrico}

		\subsection{Campo elettrico}
				\begin{equation}
				E_y= E_0e^{-j\beta z}+E_0 \Gamma e^{j \beta z}=E_0(e^{-j \beta z}+\Gamma e^{j \beta z}) =E_0(1 - \Gamma)e^{-j \beta z} + E_0 \Gamma \cos(\beta z)
				\end{equation}

		\subsection{Campo magnetico}

				\begin{equation}
				H_{max}=H^+(1+\Gamma)=\frac{E+}{\eta_{TE_{10}}}{(1+|\Gamma|)}
				\end{equation}

		\subsection{Frequenza di work}

				\begin{equation}
				f_w=\frac{\frac{c}{2a}+\frac{c}{2b}}{2}
				\end{equation}

		\subsection{Frequenza di cut}
	
				La frequenza di cut è la frequenza limite di funzionamento monomodale
				\begin{equation}
				f_c=\frac{c}{2a}
				\end{equation}
				Dove $a$ è il lato piu lungo della guida.

		\subsection{Lunghezza d'onda in setto di dielettrico}

			\begin{equation}
			\lambda_{TE}=\frac{1}{\sqrt{\epsilon_r}}\frac{\lambda}{\sqrt{1- (\frac{f_c}{\sqrt{\epsilon_r}f_w})^2}}
			\end{equation}
			

		\subsection{Frequenza in setto di dielettrico}




		\subsection{Impedenza modale}

				\begin{equation}
				\eta_{TE}=\frac{\eta_0}{ \sqrt{ 1- ( \frac {f_{cut}} {f_{work}} )^2 } }
				\end{equation}

			\subsection{Potenza in guida}
				\begin{equation}
				P_0=\frac{E_0^2}{4\eta_{TE}}ab
				\end{equation}

\chapter{Propagazione di onde piane}

	\subsection{Campo elettrico}

		\begin{equation}
		E=H\eta
		\label{eq:campo-elettrico}
		\end{equation}
		

	\subsection{Campo magnetico}
		\begin{equation}
		H=\frac{E}{\eta}
		\label{eq:campo-magnetico}
		\end{equation}
	

	\subsection{Potenza propagante}


		\begin{equation}
		P=S*Area=\frac{EH}{2}*Area=\frac{E^2}{2\eta}=\frac{H^2\eta}{2}
		\label{eq:potenza-propagante}
		\end{equation}
	
	\subsection{Potenza dissipata}

		La potenza dissipata si calcola in un mezzo si calcola come la differenza tra la potenza finale e quella iniziale.
		\begin{equation}
			P_{diss}=P_{finale}-P_{iniziale}
			\label{eq:potenza-dissipata}
		\end{equation}
	
	\subsection{Costante di propagazione d'onda}
		\[
		\gamma=\sqrt{j\omega\mu(\sigma+j\omega\epsilon_0\epsilon_r)}
		\]
	\subsection{Impedenza intrinseca}
		\[
		\eta=\sqrt{\frac{j\omega\mu}{\sigma+j\omega\epsilon_0\epsilon_r}}
		\]
		

	\subsection{Angolo di Brewster}

	L'angolo di Brewster è l'angolo di incidenza a cui abbiamo trasmissione totale:
	\[
	\sin(\theta_B)=\sqrt{\frac{\epsilon_2}{\epsilon_2+\epsilon_1}}
	\]




	\subsection{Propagazione}
	Ricordiamo la propagazione di un onda piana in un mezzo:
	

	\[
	E(x)=E_0e^{-\gamma x}
	\]
	
	Da cui usando la ~\ref{eq:campo-magnetico} possiamo calcolare il campo magnetico, e usando la ~\ref{eq:potenza-propagante} e la ~\ref{eq:potenza-dissipata} possiamo calcolare la potenza dissipata nel volume considerato.


	\subsection{Polarizzazione}

	Conoscendo le componenti x e y del campo elettrico è possibile individuale la polarizzazione dell'onda.

	\raggedright	{1) Se i moduli della componente x e y sono diversi:}

	\begin{itemize} 
	\item \textbf{Polarizazzione ellittica}
	\end{itemize}
	\raggedright{	2) Se le due componenti sono uguali in modulo, si disegnano sul piano di Gauss i fasori corrispondenti ai campi $E_x$ ed $E_y$.}
	\begin{itemize}
	\item \textbf{Polarizzazione lineare}: se i due vettori hanno la stessa fase\footnote{Sono sovrapposti sul piano di Gauss} o sono sfasati di $\pi$.
	\item \textbf{Polarizzazione Circolare} se i due vettori hanno uno sfasamento.
	\end{itemize}
		
	Rimane ora da determinare la tipologia di polarizzazione circolare.
	
	\raggedright{3) Individuiamo il senso di rotazione:}

	\begin{itemize}

	\item Antiorario: se abbiamo un onda propagante ($e^{-\beta z}$)
	\item Orario: se abbiamo un onda antipropagante ($e^{+\beta z}$)

	\end{itemize}
	\raggedright{4) Dobbiamo far ruotare uno dei due vettori, tenendo fermo l'altro, in modo da chiudere l'angolo convesso\footnote{$<180$} formato dai due.}

	\raggedright{	5) Determiniamo ora la polarizzazione secondo il seguente schema:}

	\begin{itemize}
	\item Polarizzazione Circolare sinistra: se è il vettore $x$ a muoversi verso $y$ 
	\item Polarizzazione Circolare destra: se è il vettore $y$ a muoversi verso $x$ 
	\end{itemize}	


	\section{Riconoscimento onde TEM}
	Affinchè un'onda possa essere chiamate onda TEM devono verificarsi le seguenti:

	\begin{itemize}

	\item Campo elettrico e magnetico siano in fase 

	\item Campo elettrico e campo magnetico siano proporzionali come $E=\eta H$

	\item Campo elettrico e campo magnetico devono essere perpendicolari tra loro e l'onda deve propagarsi su una terza direzione a loro perpendicolare

	\end{itemize}



\chapter{Antenne}
		\textbf{Formula generale\footnote{Valida per tutti i tipi di antenne}}
		\begin{equation}
		\frac{G}{A_e}=\frac{4\pi}{\lambda^2}
		\end{equation}
	    Dove G è il guadagno che si calcola come $G=\mu D$\footnote{Dove $D=\frac{S_{reale}}{S_{isotropa}}$ è la direttività}.

	    Il
	 \section{Dipoli elettrici in trasmissione}

		\subsection{Campo elettrico trasmesso}

		Il campo elettrico radiale è un campo vicino che decade a distanze maggiori di $2\lambda$
		
		\begin{equation}
		E_r=\frac{I{l}}{2\pi}e^{-j\beta r}(\frac{ \eta_0}{R^2}+\frac{1}{j \omega \epsilon R^3})\cos(\theta)
		\end{equation}
		
		\[
		 E_{\theta} = \frac {Il} {4\pi} e^{-j\beta r} ( \frac {j\omega \mu_0} {R} + \frac {\eta_0} {R^2} + \frac {1}  {j \omega \epsilon R^3} ) \sin(\theta)
		\]

		\subsection{Campo magnetico trasmesso}

		\begin{equation}
		H_{\phi}=\frac{Il}{4\pi}e^{-j\beta r}(\frac{j \beta}{R}+\frac{1}{R^2})\sin(\theta)
		\end{equation}

		\subsection{Potenza trasmessa}

	  \[
	  P_T=\frac{1}{2}|I|^2R_r
	  \]

	  	\subsection{Potenza dissipata}

	  \[
	  P_{diss}=\frac{1}{2}|I|^2R_d
	  \]

	  \subsection{Efficienza}

	  	\[
		\nu=\frac{P_{T}}{P_{tot}}
	  	\]


		\section{Dipoli elettrici in ricezione}

			\subsection{Resistenza di radiazione}			
				\begin{equation}
				R_r=\frac{2}{3}\pi\eta_0(\frac{l_e}{\lambda})^2
				\end{equation}

			\subsection{Area equivalente}			
				\begin{equation}
				A_e=|l_e|^2\frac{\eta_0}{4R_r}
				\end{equation}

			\subsection{Lunghezza equivalente}			
				\begin{equation}
				l_e=\frac{V_0}{E_{inc}}
				\label{eq:lunghezza-equivalente}
				\end{equation}

			\subsection{Potenza disponibile}			
				\begin{equation}
			P_d=\frac{|V_0|^2}{8(R_r+R_d)}
				\end{equation}
			Dove:

			\begin{itemize}
			\item $R_r$ è la resistenza di radiazione.
			\item $R_d$ è la resistenza interna.
			\end{itemize}
			
			\subsection{Densità di potenza incidente}

			\[
			S_{inc}=\frac{|E_{inc}|^2}{\eta_0}	
			\]

			\subsection{Tensione a vuoto}			
				Dalla ~\ref{eq:lunghezza-equivalente} otteniamo il valore della tensione ai capi dell'antenna:
				\begin{equation}
				V_0=E_{inc}l_e
				\end{equation}



			

		\section{Dipoli magnetici}
			
			\subsection{Campo elettrico trasmesso}

			\subsection{Campo magnetico trasmesso}

		\section{Spira}

			\subsection{Campo elettrico}

				\begin{equation}
				E_\phi=\frac{j\omega\mu I*Area}{4\pi}\frac{j\beta}{R}\sin(\theta)e^{-j\beta R}
				\end{equation}

		\section{Solenoide ricevente}
			
			Possiamo modellizzare il solenoide come la serie di una resistenza e di una induttanza.

			Ricordiamo dalla ~\ref{eq:induttanza-solenoide} l'induttanza del solenoide.

		 	\subsection{Tensione a vuoto}

				 \[
				 V_0=j\omega\mu_0\mu_rH_{\perp}N*Area
				 \]

			\subsection{Densità di potenza}

			 	\[
			 	S_{inc}=\frac{HE}{2}=\frac{H^2\eta_0}{2}
			 	\]

			 	%Da cui ricordando la
			\subsection{Resistenza di radiazione}
				 Mentre la resistenza di radiazione viene calcolata come:

				 \[
				 R_{r_{sol}}=N^2R_{r_{R}}=N^2\eta_0\frac{8\pi^3}{3}(\frac{Area}{\lambda^2})^2
				 \]





\end{document}