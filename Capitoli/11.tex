\chapter{Antenne}
		\textbf{Formula generale\footnote{Valida per tutti i tipi di antenne.}}
		\begin{equation}
		\frac{G}{A_e}=\frac{4\pi}{\lambda^2}
		\end{equation}
	    Dove G è il guadagno che si calcola come $G=\mu D$\footnote{Dove $D=\frac{S_{reale}}{S_{isotropa}}$ è la direttività.}.

	    Il
	 \section{Dipoli elettrici in trasmissione}

		\subsection{Campo elettrico trasmesso}

		Il campo elettrico radiale è un campo vicino che decade a distanze maggiori di $2\lambda$:
		
		\begin{equation}
		E_r=\frac{I{l}}{2\pi}e^{-j\beta R}(\frac{ \eta_0}{R^2}+\frac{1}{j \omega \epsilon R^3})\cos(\theta)
		\end{equation}
		
		\begin{equation}
		 E_{\theta} = \frac {Il} {4\pi} e^{-j\beta R} ( \frac {j\omega \mu_0} {R} + \frac {\eta_0} {R^2} + \frac {1}  {j \omega \epsilon R^3} ) \sin(\theta)
		\end{equation}

		\subsection{Campo magnetico trasmesso}

		\begin{equation}
		H_{\phi}=\frac{Il}{4\pi}e^{-j\beta R}(\frac{j \beta}{R}+\frac{1}{R^2})\sin(\theta)
		\end{equation}
		\subsection{Resistenza di radiazione}			
				\begin{equation}
				R_r=\frac{2}{3}\pi\eta_0(\frac{l_e}{\lambda})^2
				\end{equation}



		\subsection{Potenza trasmessa}

	  \begin{equation}
	  P_T=\frac{1}{2}|I|^2R_r
	  \end{equation}

	  	\subsection{Potenza dissipata}

	  \begin{equation}
	  P_{diss}=\frac{1}{2}|I|^2R_d
	  \end{equation}

	  \subsection{Efficienza}

	  	\begin{equation}
		\nu=\frac{P_{T}}{P_{tot}}
	  	\end{equation}


		\section{Dipoli elettrici in ricezione}

			\subsection{Resistenza di radiazione}			
				\begin{equation}
				R_r=\frac{2}{3}\pi\eta_0(\frac{l_e}{\lambda})^2
				\end{equation}

			\subsection{Area equivalente}			
				\begin{equation}
				A_e=|l_e|^2\frac{\eta_0}{4R_r}
				\end{equation}

			\subsection{Lunghezza equivalente}			
				\begin{equation}
				l_e=\frac{V_0}{E_{inc}}
				\label{eq:lunghezza-equivalente}
				\end{equation}

			\subsection{Potenza disponibile}			
				\begin{equation}
						P_d=\frac{|V_0|^2}{8(R_r+R_d)}
				\end{equation}
			Dove:

			\begin{itemize}
			\item $R_r$ è la resistenza di radiazione.
			\item $R_d$ è la resistenza interna.
			\end{itemize}
			
			\subsection{Densità di potenza incidente}

			\begin{equation}
			S_{inc}=\frac{|E_{inc}|^2}{\eta_0}	
			\end{equation}

			\subsection{Tensione a vuoto}			
				Dalla ~\ref{eq:lunghezza-equivalente} otteniamo il valore della tensione ai capi dell'antenna:
				\begin{equation}
				V_0=E_{inc}l_e
				\end{equation}

		\section{Spira}

			\subsection{Campo elettrico}

				\begin{equation}
				E_\phi=\frac{-j\omega\mu I*Area}{4\pi}\frac{j\beta}{R}\sin(\theta)e^{-j\beta R}
				\end{equation}

		\section{Solenoide ricevente}
			
			Possiamo modellizzare il solenoide come la serie di una resistenza e di una induttanza.

			Ricordiamo dalla ~\ref{eq:induttanza-solenoide} l'induttanza del solenoide.

		 	\subsection{Tensione a vuoto}

				 \begin{equation}
				 V_0=j\omega\mu_0\mu_rH_{\perp}N*Area
				 \end{equation}

				 Dove:
				 \begin{itemize}
				 \item N è il numero di spire del solenoide 
				 \item $H_{\perp}$ è il campo magnetico incidente in direzione assiale rispetto al solenoide
				 \end{itemize}

			\subsection{Densità di potenza}
				Come già visto nella \ref{eq:potenza-propagante}:
			 	
			 	\begin{equation}
			 	S_{inc}=\frac{HE}{2}=\frac{H^2\eta_0}{2}
			 	\end{equation}

			\subsection{Resistenza di radiazione}
				 Mentre la resistenza di radiazione viene calcolata come:

				 \begin{equation}
				 R_{r_{sol}}=N^2R_{r_{R}}=N^2\eta_0\frac{8\pi^3}{3}(\frac{Area}{\lambda^2})^2
				 \end{equation}