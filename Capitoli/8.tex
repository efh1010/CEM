\chapter{Propagazione in linea}



	\section{Potenza}
		In caso di adattamento abbiamo che tutta la potenza disponibile è trasferita al carico e in particolare vale:
			
			\begin{equation}
			P_L=P_d=(\frac{V_g}{2})^2\frac{1}{2Z_g}=\frac{V_g^2}{8Z_g}
			\end{equation}

			La potenza trasmessa al carico può anche essere calcolata tramite il vettore di pointing e il coefficiente di riflessione:

			\begin{equation}
			S_{tr}=S_{inc}(1-|\Gamma|^2)
			\end{equation}

			Analogo discorso si può fare con la potenza.
			La potenza trasferita al carico risulta essere:

			\begin{equation}
			P_{load}=P_{inc}(1-|\Gamma|^2)
			\end{equation}
	\section{Tensione al carico}

			\begin{equation}
			P_L=\frac{V_{L}}{2}Re\{Y_L\}
			\label{eq:tensionealcarico}
			\end{equation}

			Tale relazione può essere utilizzata per ricavare la tensione in ogni punto della linea considerando la potenza e l'impedenza viste ad una determinata interfaccia.
	\section{Corrente al carico}


			\begin{equation}
			P_L=\frac{I_{L}}{2}Re\{Z_L\}
			\label{eq:correntealcarico}
			\end{equation}
			Come per la \ref{eq:tensionealcarico} tale relazione può essere utilizzata in ogni punto della linea.
	\section{Legge di Ohm}

	Possiamo utilizzare la legge di Ohm per ricavare la corrente conoscendo la tensione o viceversa , utilizzando il modulo dell impedenza come segue:

	\begin{equation}
	I_L=\frac{V_L}{|Z_L|}
	\end{equation}

			Come per la \ref{eq:tensionealcarico} e la \ref{eq:correntealcarico} tale relazione può essere utilizzata in ogni punto della linea.


	\section{Attenuazione}

		Se in linea abbiamo delle perdite e conosciamo il coefficiente di attenuazione $\alpha$, sappiamo che la potenza si propaga come:
		\begin{equation}
		P(x)=P_0e^{-2 \alpha x}
		\end{equation}

		Si ricordi come in presenza di perdite in linea la potenza erogata risulta uguale alla somma della potenza dissipata lungo la linea e alla potenza trasferita al carico:

		\begin{equation}
		P_{erogata}=P_{diss}+P_{load}
		\end{equation}

	\section{Rapporto d'onda stazionaria}

	\begin{equation}
	ROS=\frac{|V_{max}|}{|V_{min}|}=\frac{|V_o^+|(1+|\Gamma|)}{|V_o^+|(1-|\Gamma|)}
	\end{equation}

	\section{Linee attenuative}

		\subsection{Coefficiente di riflessione al carico}

		\begin{equation}
		\Gamma_L=\frac{Z_L-Z_C}{Z_C+Z_L}
		\end{equation}



		\subsection{Coefficiente di riflessione attenuato}
		Lungo la linea con perdite il coefficiente di riflessione viene attenuato come:
		\begin{equation}
		\Gamma_{attenuato}=\Gamma_L e^{-2\alpha L}
		\end{equation}
		Riportando sulla carta di Smith il coefficiente di riflessione attenuato possiamo ottenere l'impedenza vista all'interfaccia di ingresso, da cui è possibile ricavare il coefficiente di riflessione in ingresso:

		\begin{equation}
		\Gamma_{in}=\frac{Z_{in}-Z_G}{Z_G+Z_{in}}
		\end{equation}

		Da cui conoscendo la potenza incidente possiamo ricavare la potenza immessa in linea come:

		\begin{equation}
		P_{in}=P_d(1-|\Gamma_{in}|^2)
		\end{equation}

		\subsection{Tensione all'interfaccia di ingresso}

		Conoscendo l'impedenza di ingresso possiamo applicare un partitore di tensione come:

		\begin{equation}
		V_{in}=V_g|\frac{Z_{in}}{Z_{in}+Z_g}|
		\end{equation}
		\textbf{NOTA:}

		Si faccia attenzione al fatto che trattando potenze propaganti in linea è necessario considerare solo le parti reali delle impedenze o delle ammettenze.