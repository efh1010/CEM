\chapter{Metodo delle immagini}	


	\section{Piano conduttore}

	Per annulare il potenziale su un piano conduttore pongo un carica immagine simmetricamente dalla parte opposta rispetto alla carica.

	\section{Linea di trasmissione}

	Esiste una soluzione unica se l'angolo tra i due piani coduttori della linea è$\frac{\pi}{n}$ con n intero.
	Verificata questa condizione dovrò porre $2n-1$ cariche immagine simmetricamente rispetto al sistema.

	\section{Sfera}

	Per annulare il potenziale su una sfera all esterno della quale è posta una carica devo porre una carica immagine a distanza $d'$ dal centro lungo la congiungente tra il centro e la carica esterna tali che:

	\begin{equation}
	d'=\frac{R^2}{d}
	\end{equation}
	Analogamente viene trattato il caso di conduttori cilindrici con distribuzioni di carica lineari.
	\\

	Utilizzando la ~\ref{eq:potenzialecaricalineare} e la ~\ref{eq:potenzialecaricapuntiforme} e la sovrapposizione degli effetti è quindi possibile calcolare il potenziale tra i conduttori in funzione della carica. Da cui ricordando la relazione $C=\frac{Q}{V}$ è possibile calcolare la capacità della linea e quindi la sua impedenza caratteristica.