
\chapter{Distribuzioni di carica}
\label{cap:capitolo1}


	\section{Conduttori Coassiali}
		\subsection{Tensione tra i conduttori}

		La tensione tra due conduttori coassiali riempiti con dielettrico , sui quali è distribuita una densità di carica superficiale $\rho_l$ opposta si calcola come:
		 \begin{equation}
		 V= \frac {\rho_l}  {2 \pi \epsilon_0} ln ( \frac { r_{out} } { r_{in} } )
		 \end{equation}
	
		\subsection{Campo elettrico}

		Il campo elettrico in coassiale in funzione del raggio si puo ricavare come:
		\begin{equation}
		|E|= \frac {|V|} {ln (\frac {r_{in}} {r_{out}} )} \frac {1} {r} = - \frac{\rho}{2 \pi \epsilon_0 \epsilon_r r}
		\label{eq:campoeletricocoassiale}
		\end{equation}


	\section{Conduttori cilindrici affiancati}

		Il potenziale tra due conduttori cilindrici affiancati su cui è distribuita una densità di carica $\rho_l$ opposta risulta:

		\begin{equation}
		V=\frac{\rho_l}{2 \pi \epsilon_0}{ln(\frac{d^2}{R^+R^-})}
		\label{eq:potenzialebifilari}
		\end{equation}

		La capacità risulta invece:

		\begin{equation}
		C=\frac{2 \pi \epsilon_0}{ln(\frac{d^2}{R^+R^-})}
		\end{equation}

	\section{Distribuzioni di carica lineare}

		\subsection{Potenziale dovuto a distribuzioni di carica lineari}
		E' possibile calcolare il potenziale in punto qualsiasi dello spazio come: 
		\begin{equation}
		V=\frac{\rho}{2\pi \epsilon}ln(\frac{d_{punto-carica}}{d_{carica-origine}})
		\label{eq:potenzialecaricalineare}
		\end{equation}

	\section{Cariche puntiformi}

		Il potenziale in un punto P generato da un carica puntiforme, dato dall' integrazione del campo elettrico coulombiano, risulta:

		\begin{equation}
		V(P)=\frac{Q}{4\pi \epsilon r}
		\label{eq:potenzialecaricapuntiforme}
		\end{equation}
		Dove:
		\begin{itemize}
		\item "Q" è la carica puntiforme.

		\item "r" è la distanza del punto P\footnote{In cui calcoliamo il potenziale} dalla carica.
		\end{itemize}
