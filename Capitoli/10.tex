
\chapter{Propagazione di onde piane}

	\subsection{Campo elettrico}
	Usando la definizione di impedenza intrinseca si ricava la relazione tra campo elettrico e magnetico in un mezzo:
		\begin{equation}
		E=H\eta
		\label{eq:campo-elettrico}
		\end{equation}
		

	\subsection{Campo magnetico}
	Analogamente alla ~\ref{eq:campo-elettrico} si ha: 
		\begin{equation}
		H=\frac{E}{\eta}
		\label{eq:campo-magnetico}
		\end{equation}
	

	\subsection{Potenza propagante}


		\begin{equation}
		P=S*Area=\frac{EH}{2}*Area=\frac{E^2}{2\eta}*Area=\frac{H^2\eta}{2}*Area
		\label{eq:potenza-propagante}
		\end{equation}
	
	\subsection{Potenza dissipata}

		La potenza dissipata in un mezzo si calcola come la differenza tra la potenza finale e quella iniziale.
		\begin{equation}
			P_{diss}=P_{finale}-P_{iniziale}
			\label{eq:potenza-dissipata}
		\end{equation}
	
	\subsection{Costante di propagazione d'onda}
		\begin{equation}
		\gamma=\sqrt{j\omega\mu(\sigma+j\omega\epsilon_0\epsilon_r)}
		\end{equation}
	\subsection{Impedenza intrinseca}
		\begin{equation}
		\eta=\sqrt{\frac{j\omega\mu}{\sigma+j\omega\epsilon_0\epsilon_r}}
		\end{equation}
		

	\subsection{Angolo di Brewster}

	L'angolo di Brewster è l'angolo di incidenza a cui abbiamo trasmissione totale:
	\begin{equation}
	\sin(\theta_B)=\sqrt{\frac{\epsilon_2}{\epsilon_2+\epsilon_1}}
	\label{eq:brewster}
	\end{equation}




	\subsection{Propagazione}
	Ricordiamo la propagazione di un onda piana in un mezzo:
	

	\begin{equation}
	E(x)=E_0e^{-\gamma x}
	\end{equation}
	
	Da cui usando la ~\ref{eq:campo-magnetico} possiamo calcolare il campo magnetico, e usando la ~\ref{eq:potenza-propagante} e la ~\ref{eq:potenza-dissipata} possiamo calcolare la potenza dissipata nel volume considerato.


	\subsection{Polarizzazione}

	Conoscendo le componenti x e y del campo elettrico è possibile individuale la polarizzazione dell'onda.

	\raggedright	{1) Se i moduli della componente x e y sono diversi\footnote{A meno che le due componenti non abbiamo sfasamento o nullo o uguale a $\pi$.}:}

	\begin{itemize} 
	\item \textbf{Polarizazzione ellittica}
	\end{itemize}
	\raggedright{	2) Se le due componenti sono uguali in modulo, si disegnano sul piano di Gauss i fasori corrispondenti ai campi $E_x$ ed $E_y$.}
	\begin{itemize}
	\item \textbf{Polarizzazione lineare}: se i due vettori hanno la stessa fase\footnote{Sono sovrapposti sul piano di Gauss} o sono sfasati di $\pi$.
	\item \textbf{Polarizzazione Circolare} se i due vettori hanno uno sfasamento.
	\end{itemize}
		
	Rimane ora da determinare la tipologia di polarizzazione circolare.
	
	\raggedright{3) Individuiamo il senso di rotazione:}

	\begin{itemize}

	\item Antiorario: se abbiamo un onda propagante ($e^{-\beta z}$)
	\item Orario: se abbiamo un onda antipropagante ($e^{+\beta z}$)

	\end{itemize}
	\raggedright{4) Dobbiamo far ruotare uno dei due vettori, tenendo fermo l'altro, in modo da chiudere l'angolo convesso\footnote{$<180$} formato dai due.}

	\raggedright{	5) Determiniamo ora la polarizzazione secondo il seguente schema:}

	\begin{itemize}
	\item Polarizzazione Circolare sinistra: se è il vettore $x$ a muoversi verso $y$ 
	\item Polarizzazione Circolare destra: se è il vettore $y$ a muoversi verso $x$ 
	\end{itemize}	


	\section{Riconoscimento onde TEM}
	Affinchè un'onda possa essere chiamate onda TEM devono verificarsi le seguenti:

	\begin{itemize}

	\item Campo elettrico e magnetico siano in fase 

	\item Campo elettrico e campo magnetico siano proporzionali come $E=\eta H$

	\item Campo elettrico e campo magnetico devono essere perpendicolari tra loro e l'onda deve propagarsi su una terza direzione a loro perpendicolare

	\end{itemize}